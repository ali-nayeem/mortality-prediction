\section{Related works} \label{s:related}
\textcolor{blue}{A detail review of the related works is available in the supplementary materials. Here we describe some very relevant recent works.}

\textcolor{blue}{Calvert \textit{et al.} have developed an algorithm, \textit{AutoTriage}, which uses 8 common clinical variables (mostly from physiological measurements) and 2/3 discretized variables \cite{Calvert2016}. These clinical variables produce sub-scores. A combination of weighted sub-scores is used by them as the final score for mortality prediction of ICU patients. They have conducted experiments on Medical Information Mart for Intensive Care (MIMIC) III medical ICU dataset. Bhattacharya \textit{et al.} conducted a study for ICU mortality prediction addressing the class imbalance issue of the ICU data in the context of binary classification through a feature transformation approach \cite{Bhattacharya2017}. They used demographic data, 37 lab investigations and some physiological signal measurements for their experiments. Xie \textit{et al.} outlined the essential procedures and concepts for developing prediction models for in-hospital mortality prediction for ICU patients in \cite{Xie2017}. In this work Artificial Neural Network (ANN), Decision Tree (DT), and support vector machines (SVM) are showed promising results in analyzing large, and heterogeneous data but Logistic Regression (LR) is not as compared with others. They used physiological variables along with others. Awad \textit{et al.} proposed a method called Early Mortality Prediction for ICU patients (EMPICU) \cite{Awad2017}. The mortality was predicted 6 hours after admission of ICU patients. This study included demographic, physiological, vital signs and laboratory test variables collected from MIMIC-II database. Nguyen \textit{et al.} proposed a deep learning architecture based on Long Short-Term Memory networks (LSTM) with layered attentions mechanism to predict ICU mortality to address the issue of missing measurements \cite{Nguyen2017}. They used 41 measure types including vital signs.} 

\textcolor{blue}{Johnson \textit{et al.} conducted 38 experiments of 28 published studies that used MIMIC database attempting to reproduce the cohorts (group of variables) used in these studies in the context of performance of mortality prediction models for presenting benchmarks \cite{Johnson2nd2017}. Darabi \textit{et al.} developed a technique in which they applied Gradient Boosting Decision Trees (GBT) and deep neural networks to predict mortality of ICU patients \cite{Darabi2018}. They used grid-search to find the optimal parameters for the model. They conducted their experiments on the medical codes (diagnosis codes, procedure codes, diagnosis related group code) collected from MIMIC-III database.} 

\textcolor{blue}{Sadeghia \textit{et al.} proposed a method which utilized statistical and signal-based features for early hospital mortality prediction \cite{Sadeghi2018}. They worked with only vital signals, i.e., heart signals of the patients and these were collected from MIMIC III database. Zheng and Shi utilized LSTM and Recurrent Neural Network (RNN) based deep learning techniques for ICU mortality prediction \cite{Zheng2018}. They used a statistical approach to preprocess the data. They presented a data imputation method based on the Gaussian process. They used ICU data containing 36 variables from the PhysioNet. These 36 features are clearly mentioned in terms of importance from medical perspective.} 

\textcolor{blue}{Zahid and Joon Lee explored deep learning techniques focusing on the Self Normalizing Neural network (SNN) for predicting mortality of the ICU patients \cite{Zahid2018}. They used demographic information, vital physiological signs, progress notes by physicians and nurses, reports from imaging studies, lab test results, International Classification of Diseases-9 (ICD-9) codes, daily Simplified Acute Physiology Score (SAPS) and Sequential Organ Failure Assessment (SOFA) score, discharge summaries, ICU, hospital lengths of stay (LoS), and output of the hospital mortality. Purushotham \textit{et al.} performed a study for presenting benchmarking results for clinical prediction tasks (e.g., mortality prediction, LoS prediction, and ICD-9 code group prediction) using deep learning models, Super ICU Learner Algorithm (SICULA), and SAPS-II and SOFA \cite{Purushotham2018}. They have outlined that deep learning models are consistently doing better compared to all other approaches.} 

\textcolor{blue}{Gennatas \textit{et al.} have proposed Expert-Augmented Machine Learning (EAML) that guides the extraction of expert knowledge and it's integration into machine-learned models \cite{Gennatas2019}. They conducted experiments on MIMIC II \& MIMIC III ICU dataset for predicting mortality of the ICU patients. They used vital signs with other clinical variables in their experiments. Caicedo-Torres and Gutierrez proposed a deep multi-scale convolutional architecture trained on MIMIC III dataset for predicting mortality of ICU patients \cite{Torres2019}. They demonstrated visual explanations focusing on how the network treats these inputs as important features. They used 22 different variables (including vital signs) roughly matching the concepts used by the SAPS-II score. De Lange \textit{et al.} proposed a cumulative prognostic score model for predicting mortality of ICU patients who are older than 80 years \cite{Lange2019}. They developed multivariable LR model in which variables are selected using LASSO. Data of 306 ICU patient from 24 European countries were collected and their 24 clinical variables were used in the experiment.}

We note that, our current work is different from the research endeavours mentioned above. Most of the above works use vital signs or demographic variables, whereas only lab test results are used in our study. \textcolor{blue}{Some of the above works mainly focus on clinical notes. Some works indeed consider lab tests measurements albeit only a small set thereof and that too in combination with other features. We on the contrary consider only lab tests measurements and in fact a large set thereof. Also, some of the above studies deal with particular disease oriented data; we generally work for the mortality of ICU patients irrespective of any disease.} Additionally we have used different kinds of datasets for the experiments. More will be elaborated at later sections of this paper.

\textcolor{blue}{Some studies deal with data mining based solutions for developing clinical decision support (CDS) systems.} Herland \textit{et al.} have conducted a comprehensive survey on this topic, i.e., clinical data mining applications based on big data in health informatics \cite{Herland}. Cai \textit{et al.} have proposed an approach based on Baysian network to develop models using EHR for real time prediction of several targets, including the length of hospital stay, mortality, and readmission of hospitalized patients \cite{Cai:2016}. We note here that majority of the above studies focus on a single technique and mainly exploit data mining methods, whereas our work proposes a hybrid approach, features are selected at first and then ensemble models are generated using FVC technique with various ensemble classification techniques. Therefore it is believed that our current approach is applicable to any CDS related systems in general. 

\textcolor{blue}{Additionally, some of the above deep learning methodology based studies deal with Big data in the context of in-hospital mortality of ICU patients. According to Meyer \textit{et al.} \cite{Meyer2018} adding more variables into the actual data matrix increases the dimension of the matrix which may become computationally burdensome for this kind of models. But FVC technique \cite{mehedy-masud:2017:fvc} is reverse of this, deals with large data and has shown the improved performance on training or testing time.}

Finally, a subset of the authors were involved in proposing the FVC approach in \cite{mehedy-masud:2017:fvc} and ORCU (a novel orthogonal clustering technique) in \cite{mehedy-masud:2018:frmwrk} where some preliminary experiments and findings were discussed. In this work we proposed further improvements of the above mentioned works and conducted extensive experiments with real clinical data thereby reaching a promising milestone in this research endeavour.