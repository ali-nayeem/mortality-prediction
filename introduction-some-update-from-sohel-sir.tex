\section{Introduction}
Research on clinical decision support (CDS) and clinical prediction (CP) has been getting high attention in recent years for its significant impacts on the improvements in quality, safety, efficiency, and effectiveness in health-care domain. In reality, for some cases, patients are intensively cared and monitored by the health care providers, such as, intensive care unit (ICU) patients. Timely and effective actions recommended by CDS or CP can help care-givers to take necessary actions to avoid any unwanted circumstances or to make an improvement in health condition. Therefore, study in this domain has been gaining more and more emphasis by the researchers since the last decade.

Nowadays most of the hospitals, clinics and health institutions are deployed with various electronic health monitoring devices which are continuously producing data. Many of these medical facilities store these data in  a systematic and standard manner in the form of Electronic Health Records (EHRs). Researchers have already demonstrated much advancement on data analytics techniques. These advanced techniques, applied on EHRs, can create great improvement in CDS and CP systems with the help of high performance computing services.

Health data of our interest could be continuous, such as, vital signs or they could be discrete or incremental, such as, lab tests data. Moreover, there are other types of data available additionally, such as, records of medication, nurse notes, demographic data, administrative (e.g. admissions) and procedural (e.g. caregiver name) information. \textcolor{blue}{Among these, vital signs and lab test records have mostly been used in the literature \cite{Mao,Fialho,Baumgartner,Cheng, Ghassemi2015, Jin2018, Yoon2016, Johnson2017, Calvert2016, Suresh2017, Bhattacharya2017, Xie2017, Awad2017, Nguyen2017, Zhang2017, Johnson2nd2017, Davoodi2018, Sadeghi2018, Zheng2018, Johnson3rd2018, Zahid2018, Purushotham2018, Meyer2018, Hsieh2018, Ho2019, Gennatas2019, Torres2019}}. Using the vital signs data, some studies have illustrated the possibility to predict patient situation (e.g., deterioration) ahead of time with a goal to warn caregivers to take timely and reliable measures to save the patient’s life. Many challenges are involved in building such a system using vital signs \textcolor{blue}{as discussed in \cite{Mao,Fialho,Baumgartner,Cheng, ZhengpingChe2016, Zahid2018, Zheng2018, Xiao2018}. On the other hand, research works are relatively rare focused on lab test data. To the best of our knowledge the only work in the literature considering only lab test data are the work of  Masud \& Harahsheh \cite{mehedy-masud:2017:fvc} and Masud and Cheratta \cite{mehedy-masud:2018:frmwrk}. Hence, this study primarily focuses on lab test records}.

The lab test records are not readily usable for building a good prediction system. It contains severe challenges related to be treated as a feature. Some of these challenges are mentioned in some recent studies \cite{mehedy-masud:2017:fvc, mehedy-masud:2018:frmwrk} and are discussed as follows for the sake of completeness. The first challenge is a variable length feature vector. It occurs because each patient undergoes a different subsets (possibly overlapping) of lab tests. As a result, there is no uniform feature vector across all patients. But most learning algorithms need uniform feature vectors, which necessitates the prediction of missing values into the feature vector of each patient. The second challenge is the high dimensionality of the data due to the large number different possible lab tests that can be done on a patient. The third challenge is the class imbalance, which is also found in many problems in the medical domain. And the last but not the least challenge is the longitudinal features, i.e., a patient may undergo the same test more than once. Interestingly, the first two challenges are interrelated making the scenario more complex. For example, missing values are introduced primarily because of high dimensionality, and they bring noise, redundancy, and sparsity in the dataset. One of the general solutions in this context is to apply some kind of feature selection: features may be evaluated and examined by some standard feature ranking algorithms that give score to each feature. Clearly missing values pose an issue in this approach and need be handled properly. Additionally, class imbalance make the situation worse. In \cite{mehedy-masud:2017:fvc, mehedy-masud:2018:frmwrk}, a subset of the authors proposed a novel feature vector compaction (FVC) technique for addressing the missing values and high dimensionality problem and presented some preliminary results. The current work proposes several modifications to FVC in order to improve its efficiency and effectiveness.

The feature vector compaction (FVC) exploits a measure called vacuum count (VC). It deals with missing values found when two feature vectors are combined. In this work we consider another angle as follows. Instead of only focusing on missing values, we also focus on how many common features are available between two sets of feature vectors. Thus, we propose a new measure called common feature count (CFC) to compare two sets of feature vectors.

The main contributions of this research work are as follows. At first we propose a feature ranking based feature selection approach along with FVC technique for lab test results with a goal to improve the performance of the clinical prediction. Secondly, the proposed methodology systematically handles the missing data, high dimension, and class imbalance issues. Thirdly we explore how the proposed CFC technique can be used to improve the accuracy of the mortality prediction. And, last but not the least, we propose an ensemble approach to achieve even better results.

This paper is organized as follows. Section \ref{s:related} presents a brief literature review. Section \ref{s:methods} describes the proposed techniques in detail. Then Section \ref{s:experiments} states the experimental setup and other information related thereof. Section \ref{s:results} presents the results of the experiments over various datasets with an in-depth analysis followed by a discussion in Section \ref{s:discussions}. Finally Section \ref{s:conclusion} briefly concludes the paper with future research directions.
