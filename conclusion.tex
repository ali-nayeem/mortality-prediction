\section{Conclusion} \label{s:conclusion}
ICU patients must be closely monitored by health care providers. Early mortality prediction can help them to take proper and timely actions for these patients. The existing researches are mostly based on the vital signs. Very few studies are available in the literature on mortality prediction using lab tests data. We have proposed feature ranking based methods hybridized with other feature vector compaction (FVC) processes utilizing common feature counting (CFC) and vacuum count (VC) ensemble techniques to predict mortality more accurately using the lab tests data. 
      
To advance the current state of the art, this study proposes the feature ranking based approaches, the modification of FVC techniques and CFC as an ensemble technique. From the results of extensive experiments, it has been seen that the proposed technique outperforms other models for predicting mortality using lab test data. 

The study additionally shows how CFC can contribute in improving the performance of the model. In addition to that, it demonstrates the impact of Vertical(V)-Horizontal (H) FVC technique along with Vacuum Count (VC) for better performance. Moreover, this work also compares the performance of some standard classifier algorithms in this context, such as, J48 classifier, NaiveBayes (NB) classifier, RandomForest (RF) classifier and Support Vector Machine (SVM).           
  
In future, we plan to integrate background knowledge on data into our approach. Additionally, missing value replacement using some other methods will be examined and evaluated. The issue of longitudinal feature in the lab tests data will be addressed. For the real world works at ICU, the lab test data will be merged with vital signs to predict mortality rate more accurately as a whole. Currently, some subsets of features from the ranked features are considered for the model; more fine tuning in the feature selection processes may lead us to even better performance. Finally, these proposed models will be applied to other areas of clinical decision support systems where the data exhibits similar properties.