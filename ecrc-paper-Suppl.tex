
% Template for Elsevier CRC journal article
% version 1.1 dated 16 March 2010

% This file (c) 2010 Elsevier Ltd.  Modifications may be freely made,
% provided the edited file is saved under a different name

% This file contains modifications for Procedia Computer Science
% but may easily be adapted to other journals

% Changes since version 1.0
% - elsarticle class option changed from 1p to 3p (to better reflect CRC layout)

%-----------------------------------------------------------------------------------

%% This template uses the elsarticle.cls document class and the extension package ecrc.sty
%% For full documentation on usage of elsarticle.cls, consult the documentation "elsdoc.pdf"
%% Further resources available at http://www.elsevier.com/latex

%-----------------------------------------------------------------------------------

%%%%%%%%%%%%%%%%%%%%%%%%%%%%%%%%%%%%%%%%%%%%%%
%%%%%%%%%%%%%%%%%%%%%%%%%%%%%%%%%%%%%%%%%%%%%%
%%                                          %%
%% Important note on usage                  %%
%% -----------------------                  %%
%% This file must be compiled with PDFLaTeX %%
%% Using standard LaTeX will not work!      %%
%%                                          %%
%%%%%%%%%%%%%%%%%%%%%%%%%%%%%%%%%%%%%%%%%%%%%%
%%%%%%%%%%%%%%%%%%%%%%%%%%%%%%%%%%%%%%%%%%%%%%

%% The '3p' and 'times' class options of elsarticle are used for Elsevier CRC
\documentclass[3p,times]{elsarticle} %this is original one for 1 column
%\documentclass[5p,times,twocolumn]{elsarticle}

%% The `ecrc' package must be called to make the CRC functionality available
%\usepackage{ecrc}

%% The ecrc package defines commands needed for running heads and logos.
%% For running heads, you can set the journal name, the volume, the starting page and the authors

%% set the volume if you know. Otherwise `00'
%\volume{00}

%% set the starting page if not 1
%\firstpage{1}

%% Give the name of the journal
%\journalname{Procedia Computer Science}

%% Give the author list to appear in the running head
%% Example \runauth{C.V. Radhakrishnan et al.}
%\runauth{M. Z. Alam et al.}

%% The choice of journal logo is determined by the \jid and \jnltitlelogo commands.
%% A user-supplied logo with the name <\jid>logo.pdf will be inserted if present.
%% e.g. if \jid{yspmi} the system will look for a file yspmilogo.pdf
%% Otherwise the content of \jnltitlelogo will be set between horizontal lines as a default logo

%% Give the abbreviation of the Journal.
%\jid{procs}

%% Give a short journal name for the dummy logo (if needed)
%\jnltitlelogo{Procedia Computer Science}

%% Hereafter the template follows `elsarticle'.
%% For more details see the existing template files elsarticle-template-harv.tex and elsarticle-template-num.tex.

%% Elsevier CRC generally uses a numbered reference style
%% For this, the conventions of elsarticle-template-num.tex should be followed (included below)
%% If using TeX, use the style file elsarticle-num.bst

%% End of ecrc-specific commands
%%%%%%%%%%%%%%%%%%%%%%%%%%%%%%%%%%%%%%%%%%%%%%%%%%%%%%%%%%%%%%%%%%%%%%%%%%

%% The amssymb package provides various useful mathematical symbols
\usepackage{amssymb}
%% The amsthm package provides extended theorem environments
\usepackage{amsthm}
\usepackage{amsmath}

%% The lineno packages adds line numbers. Start line numbering with
%% \begin{linenumbers}, end it with \end{linenumbers}. Or switch it on
%% for the whole article with \linenumbers after \end{frontmatter}.
\usepackage{lineno}

%% natbib.sty is loaded by default. However, natbib options can be
%% provided with \biboptions{...} command. Following options are
%% valid:

%%   round  -  round parentheses are used (default)
%%   square -  square brackets are used   [option]
%%   curly  -  curly braces are used      {option}
%%   angle  -  angle brackets are used    <option>
%%   semicolon  -  multiple citations separated by semi-colon
%%   colon  - same as semicolon, an earlier confusion
%%   comma  -  separated by comma
%%   numbers-  selects numerical citations
%%   super  -  numerical citations as superscripts
%%   sort   -  sorts multiple citations according to order in ref. list
%%   sort&compress   -  like sort, but also compresses numerical citations
%%   compress - compresses without sorting
%%
%% \biboptions{comma,round}

% \biboptions{}

% if you have landscape tables
\usepackage[figuresright]{rotating}

% put your own definitions here:
%   \newcommand{\cZ}{\cal{Z}}
%   \newtheorem{def}{Definition}[section]
%   ...

\usepackage{hyperref}
\usepackage{multirow}
\usepackage{color}
\usepackage{xcolor-patch}
\usepackage{setspace}
\doublespacing

% *** SPECIALIZED LIST PACKAGES ***
%
\usepackage{algorithmic, algorithm}
% algorithmic.sty was written by Peter Williams and Rogerio Brito.
% This package provides an algorithmic environment fo describing algorithms.
% You can use the algorithmic environment in-text or within a figure
% environment to provide for a floating algorithm. Do NOT use the algorithm
% floating environment provided by algorithm.sty (by the same authors) or
% algorithm2e.sty (by Christophe Fiorio) as the IEEE does not use dedicated
% algorithm float types and packages that provide these will not provide
% correct IEEE style captions. The latest version and documentation of
% algorithmic.sty can be obtained at:
% http://www.ctan.org/pkg/algorithms
% Also of interest may be the (relatively newer and more customizable)
% algorithmicx.sty package by Szasz Janos:
% http://www.ctan.org/pkg/algorithmicx

% *** PDF, URL AND HYPERLINK PACKAGES ***
%
\usepackage{url}
% url.sty was written by Donald Arseneau. It provides better support for
% handling and breaking URLs. url.sty is already installed on most LaTeX
% systems. The latest version and documentation can be obtained at:
% http://www.ctan.org/pkg/url
% Basically, \url{my_url_here}.
\usepackage{pdfpages}

% add words to TeX's hyphenation exception list
%\hyphenation{author another created financial paper re-commend-ed Post-Script}

% correct bad hyphenation here
\hyphenation{op-tical net-works semi-conduc-tor}

% declarations for front matter

%\doublespace
\begin{document}
	
	\begin{frontmatter}
		
		%% Title, authors and addresses
		
		%% use the tnoteref command within \title for footnotes;
		%% use the tnotetext command for the associated footnote;
		%% use the fnref command within \author or \address for footnotes;
		%% use the fntext command for the associated footnote;
		%% use the corref command within \author for corresponding author footnotes;
		%% use the cortext command for the associated footnote;
		%% use the ead command for the email address,
		%% and the form \ead[url] for the home page:
		%%
		%% \title{Title\tnoteref{label1}}
		%% \tnotetext[label1]{}
		%% \author{Name\corref{cor1}\fnref{label2}}
		%% \ead{email address}
		%% \ead[url]{home page}
		%% \fntext[label2]{}
		%% \cortext[cor1]{}
		%% \address{Address\fnref{label3}}
		%% \fntext[label3]{}
		
		%\dochead{}
		%% Use \dochead if there is an article header, e.g. \dochead{Short communication}
		
		\title{Supplementary Materials for the manuscript titled ``Feature Ranking based Ensemble Classifiers for survivability prediction of ICU patients using Lab Test data''}
		%\tnotetext[mytitlenote]{Full document is available in on\href{http://www.ctan.org/tex-archive/macros/latex/contrib/elsarticle}.}
		
		%% use optional labels to link authors explicitly to addresses:
		%% \author[label1,label2]{<author name>}
		%% \address[label1]{<address>}
		%% \address[label2]{<address>}
		
		%% Group authors per affiliation:
		%% include affiliations in footnotes:
		\author[buetaddress]{Md. Zahangir Alam\fnref{myfootnote}}
		\ead{zahangirbd@gmail.com}
		
		\author[uaeuaddress]{Mohammad M. Masud \corref{mycorrespondingauthor}}
		\ead{m.masud@uaeu.ac.ae}
        
        \author[buetaddress]{M. Saifur Rahman}
		\ead{mrahman@cse.buet.ac.bd}
		
		\author[uaeuaddress]{Muhsin Cheratta}
		\ead{muhsin.j@uaeu.ac.ae}
		
		\author[buetaddress]{Muhammad Ali Nayeem\fnref{myfootnote}}			
		\ead{ali\_nayeem@cse.buet.ac.bd}			
					
		\author[buetaddress]{M. Sohel Rahman}
        \cortext[mycorrespondingauthor]{Corresponding author. E-mail: \url{m.masud@uaeu.ac.ae}}		
        \ead{msrahman@cse.buet.ac.bd}
		
		%\address[buetaddress]{Department of Computer Science and Engineering (CSE), ECE Building, West Palashi, Bangladesh University of Engineering and Technology (BUET), Dhaka-1000, Bangladesh.}
		
		\address[buetaddress]{Department of CSE, BUET, ECE Building, West Palasi, Dhaka-1205, Bangladesh.}
		\address[uaeuaddress]{College of Information Technology, United Arab Emirates University, Al Ain, UAE.}
		\fntext[myfootnote]{Supported by an ICT (PhD) Fellowship}
		
	\end{frontmatter}
	
	%%
	%% Start line numbering here if you want
	%%
	% \linenumbers
	
	%% main text
	
	\section{Overview of supplementary materials}
This supplementary file lists the following items.  
\begin{enumerate}
	\item Related works: It contains the detail literature review. 
	\item Lab tests as features for MIMIC-II datasets: It contains all the lab tests as features for MIMIC-II datasets. 
	\item Lab tests as features for MIMIC-III datasets: It contains all the lab tests as features for MIMIC-III datasets.					
	\item Results in tabular format: It contains the results of the experiments for each dataset in tabular form. 				
	\item Top ten ranked features: It contains top ten ranked features of each dataset of the proposed models.
	\item Feature wise linear regression: It contains the results of correlation analysis of some top ranked features and the patient status (alive/dead) 
\end{enumerate}

\section{Related works} \label{s:related}
Mainly, two broad categories of works related to our task at hand can be found in the literature. \textcolor{blue} {In the first category, the studies deal with ICU patients for various clinical and healthcare support systems, whereas, other category of related work addresses machine learning in general for medical decision support. A detail review of these works is available in the supplementary materials. Here we describe some relevant recent works.}

Mao \textit{et al.} have presented a data-mining approach to predict deterioration of patients in the ICU \cite{Mao}. They have used time-series data observed from different sensors attached to patient’s body, such as blood pressure, heart rate, $O_2$ saturation and so on. These time-series data are pre-processed and several features are extracted therefrom. Finally, the different feature selection approaches and optimization techniques are applied to build a prediction model. Fialho \textit{et al.} have presented a feature selection method to find the best features in order to predict ICU readmissions \cite{Fialho}. On the other hand, Baumgartner \textit{et al.} have examined the high false alarm problem in ICU monitors and have proposed an effective data mining based solution to reduce those false alarms \cite{Baumgartner}. Cismondi \textit{et al.} have considered a different dimension with respect to ICU patients: they aimed to reduce unnecessary lab testing in the ICU \cite{Cismondi}. However, they only focused on gastrointestinal bleeding. Cheng \textit{et al.} have introduced a clinical decision support (CDS) system based on association rule mining that can find associations among various variables, such as, patients’ conditions, length of ICU stays and so on. 

\textcolor{blue}{Ghassemi \textit{et al.} introduced multivariate time series modeling with the multi-task Gaussian process (GP) by conducting experiments on physiological signals and clinical notes \cite{Ghassemi2015}. For physiological signals, they used the datasets of Traumatic Brain Injury (TBI) patients. They predicted mortality using clinical notes that were collected from Medical Information Mart for Intensive Care (MIMIC) II dataset. Marafino \textit{et al.} developed elastic-net regularization oriented stochastic gradient descent-based classifiers \cite{Marafino2015} and conducted experiments on the nursing notes collected from MIMIC-II database. This study narrated some features of Least Absolute Shrinkage and Selection Operator (LASSO) and elastic-net approaches but finally used the latter approach. Grnarova \textit{et al.} proposed a convolutional document embedding approach for ICU mortality prediction using unstructured clinical notes collected from MIMIC-III database \cite{Grnarova2016}. Jin \textit{et al.} developed a multimodal neural network to jointly train time series signals and unstructured clinical text focusing on exploring the contribution of clinical text in a clinical predictive learning task \cite{Jin2018}. For predicting hospital mortality risk for ICU patients their model outperforms the benchmark. They have used vital signs with clinical text in their experiments. These studies mainly focus on clinical notes for building predictive models}.  

\textcolor{blue}{Kamkar \textit{et al.} performed a research work for finding stable feature selection using \textit{Tree-Lasso} for clinical prediction \cite{Kamkar2015} by addressing one of the problems of LASSO. In LASSO one feature of many correlated features are selected randomly, therefore, it hinders for finding a stable feature set. Baalachandran \textit{et al.} performed a retrospective study to determine the risk factors for mortality of the ICU patients with viral pneumonia \cite{Baalachandran2015}. They used LASSO regression as a selection operator. Pirracchio \textit{et al.} introduced a learning algorithm to predict mortality of ICU patients and successfully used that in real hospitals \cite{Pirracchio2015}. Gholipour \textit{et al.} leveraged a neural network to predict survival and length of stay (LoS) in the ward and the intensive care unit (ICU) of trauma patients \cite{Gholipour2015}.}  

\textcolor{blue}{Yoon \textit{et al.} have proposed an approach, named as \textit{ForecastICU}, which monitors the physiological data streams of ICU patients in real-time and generates a prompt alarm whenever the predictor hits a predefined threshold \cite{Yoon2016}. They have used a dataset obtained from UCLA Ronald Reagan Medical Center. A total of 18 temporal physiological data streams have been used in their experiments. Logistic regression (LR), support vector machines (SVM) with radial based kernel (RBK), regularized logistic regression with LASSO and random forest have been used as the classifiers in \cite{Yoon2016}. Johnson \textit{et al.} performed a study to predict real-time mortality in the ICU \cite{Johnson2017}. They collected ICU adult data for surgical, medical, neurological, coronary critical illness. They used vital signs along with other clinical variables. Their used models are Logistic regression (LR), LR with an L1 regularization penalty (L1) using the LASSO, LR with an L2 regularization penalty (L2), and Gradient Boosting Decision Trees (GBT). Here GBT performs well compared to others. But these studies primarily worked on developing models for mortality predictions in real-time}.  
 
\textcolor{blue}{Calvert \textit{et al.} have developed an algorithm, \textit{AutoTriage}, which uses 8 common clinical variables (mostly from physiological measurements) and 2/3 discretized variables \cite{Calvert2016}. These clinical variables produce sub-scores. A combination of weighted sub-scores is used by them as the final score for mortality prediction of ICU patients. They have conducted experiments on MIMIC III medical ICU dataset. Che \textit{et al.} introduced knowledge-distillation approach which used GBT to learn interpretable models \cite{ZhengpingChe2016}. They conducted experiments on Pediatric ICU dataset for acute lung injury (ALI). Suresh \textit{et al.} performed a study focusing on learning rich representation of ICU data (vitals, labs, notes, demographics) to predict clinical invasive interventions \cite{Suresh2017}. They conducted experiments on MIMIC III database using Long Short-Term Memory networks (LSTM) and Convolutional Neural Networks (CNN). Bhattacharya \textit{et al.} conducted a study for ICU mortality prediction addressing the class imbalance issue of the ICU data in the context of binary classification through a feature transformation approach \cite{Bhattacharya2017}. They used demographic data, 37 lab investigations and some physiological signal measurements for their experiments. Xie \textit{et al.} outlined the essential procedures and concepts for developing prediction models for in-hospital mortality prediction for ICU patients in \cite{Xie2017}. In this work Artificial Neural Network (ANN), Decision Tree (DT), and SVM are showed promising results in analyzing large, and heterogeneous data but LR is not as compared with others. They used physiological variables along with others. Awad \textit{et al.} proposed a method called Early Mortality Prediction for ICU patients (EMPICU) \cite{Awad2017}. The mortality was predicted 6 hours after admission of ICU patients. This study included demographic, physiological, vital signs and laboratory test variables collected from MIMIC-II database. Nguyen \textit{et al.} proposed a deep learning architecture based on LSTM with layered attentions mechanism to predict ICU mortality to address the issue of missing measurements \cite{Nguyen2017}. They used 41 measure types including vital signs.} 

\textcolor{blue}{Mikacenic \textit{et al.} developed a two-biomarker model with primary focus on in-patient mortality at Day 28 after admission of ICU patients with Systemic Inflammatory Response Syndrome (SIRS) and sepsis (life-threatening illness) \cite{Mikacenic2017}. They used LASSO as a selection operator in their study. Zhang \textit{et al.} developed a severity score for the patients with severe sepsis through conducting experiments using clinical and laboratory variables collected from MIMIC-III database \cite{Zhang2017}. They used the LASSO technique for variable selection. These studies deal with the data of a particular disease (e.g., sepsis) oriented patient in the ICU.} 

\textcolor{blue}{Kamio \textit{et al.} performed a systematic review on the use of machine learning techniques to predict clinical deterioration of the critically ill patients \cite{Kamio:2017}. They have used vital signs and other clinical variables in their experiments. Johnson \textit{et al.} conducted 38 experiments of 28 published studies that used MIMIC database attempting to reproduce the cohorts (group of variables) used in these studies in the context of performance of mortality prediction models for presenting benchmarks \cite{Johnson2nd2017}.}
 
\textcolor{blue}{Darabi \textit{et al.} developed a technique in which they applied GBT and deep neural networks to predict mortality of ICU patients \cite{Darabi2018}. They used grid-search to find the optimal parameters for the model. They conducted their experiments on the medical codes (diagnosis codes, procedure codes, diagnosis related group code) collected from MIMIC-III database. Davoodi \textit{et al.} proposed Deep Rule-Based Fuzzy System (DRBFS) to predict accurate in-hospital mortality of ICU patients \cite{Davoodi2018}. The model is capable of dealing with big data with heterogeneous mixed categorical and numeric variables. They used a heterogeneous feature set (including vital signs) of the first 48 h from ICU admissions from MIMIC-III database.} 

\textcolor{blue}{Sadeghia \textit{et al.} proposed a method which utilized statistical and signal-based features for early hospital mortality prediction \cite{Sadeghi2018}. They worked with only vital signals, i.e., heart signals of the patients and these were collected from MIMIC-III database. Zheng and Shi utilized LSTM and Recurrent Neural Network (RNN) based deep learning techniques for ICU mortality prediction \cite{Zheng2018}. They used a statistical approach to preprocess the data. They presented a data imputation method based on the Gaussian process. They used ICU data containing 36 variables from the PhysioNet. These 36 features are clearly mentioned in terms of importance from medical perspective.} 

\textcolor{blue}{Johnson \textit{et al.} attempted to develop predictive models for clinical data from multiple sources \cite{Johnson3rd2018}. They worked with publicly available eICU Collaborative Research Database. They used physiologic and laboratory measurements in their experiments. A total of 82 features were used in the study. Zahid and Joon Lee explored deep learning techniques focusing on the Self Normalizing Neural network (SNN) for predicting mortality of the ICU patients \cite{Zahid2018}. They used demographic information, vital physiological signs, progress notes by physicians and nurses, reports from imaging studies, lab test results, International Classification of Diseases-9 (ICD-9) codes, daily Simplified Acute Physiology Score (SAPS) and Sequential Organ Failure Assessment (SOFA) score, discharge summaries, ICU, hospital lengths of stay (LoS), and output of the hospital mortality. Purushotham \textit{et al.} performed a study for presenting benchmarking results for clinical prediction tasks (e.g., mortality prediction, LoS prediction, and ICD-9 code group prediction) using deep learning models, Super ICU Learner Algorithm (SICULA), and SAPS-II and SOFA \cite{Purushotham2018}. They have outlined that deep learning models are consistently doing better compared to all other approaches.} 

\textcolor{blue}{Meyer \textit{et al.} executed a study by applying RNN to predict severe complications in post cardiosurgical care in real-time \cite{Meyer2018}. Hsieh \textit{et al.} compared performances of various machine learning models and conventional scoring systems to predict the mortality of unplanned extubation patients in the ICU \cite{Hsieh2018}.}

\textcolor{blue}{Ho \textit{et al.} developed a technique, learned binary masks (LBM), to interpret a recurrent neural network (RNN) to predict a child's mortality in ICU using multi-modal time series data \cite{Ho2019}. LBM was used to identify the most noticeable features across a RNN based model. They used physiologic observations, laboratory results, drugs, and interventions (e.g., intubation parameters) over their ICU stays collected from Pediatric Intensive Care Unit (PICU) of a tertiary hospital. Pan \textit{et al.} proposed a self-correcting deep learning prediction technique by conducting experiments on the data of acute kidney injury (AKI) patients \cite{Pan2019}.} 

\textcolor{blue}{Gennatas \textit{et al.} have proposed Expert-Augmented Machine Learning (EAML) that guides the extraction of expert knowledge and it's integration into machine-learned models \cite{Gennatas2019}. They conducted experiments on MIMIC-II \& MIMIC III ICU dataset for predicting mortality of the ICU patients. They used vital signs with other clinical variables in their experiments. Caicedo-Torres and Gutierrez proposed a deep multi-scale convolutional architecture trained on MIMIC-III dataset for predicting mortality of ICU patients \cite{Torres2019}. They demonstrated visual explanations focusing on how the network treats these inputs as important features. They used 22 different variables (including vital signs) roughly matching the concepts used by the SAPS-II score.}

\textcolor{blue}{De Lange \textit{et al.} proposed a cumulative prognostic score model for predicting mortality of ICU patients who are older than 80 years \cite{Lange2019}. They developed multivariable LR model in which variables are selected using LASSO. Data of 306 ICU patient from 24 European countries were collected and their 24 clinical variables were used in the experiment. Bosma \textit{et al.} developed a prognostic multivariable model in patients discharged from the ICU to predict who was at increased risk for Potentially Harmful Medication Transfer Errors (PH-MTE) \cite{Bosma2019}. They used LASSO regression as the predictor.}

We note that, our current work is different from the research endeavours mentioned above. Most of the above works use vital signs or demographic variables, whereas only lab test results are used in our study. \textcolor{blue}{Some of the above works mainly focus on clinical notes. Some works indeed consider lab tests measurements albeit only a small set thereof and that too in combination with other features. We on the contrary consider only lab tests measurements and in fact a large set thereof. Also, some of the above studies deal with particular disease oriented data; we generally work for the mortality of ICU patients irrespective of any disease.} Additionally we have used different kinds of datasets for the experiments. More will be elaborated at later sections of this paper.

The second category of works from the literature, in general deal with data mining based solutions for developing CDS systems. Celi \textit{et al.} have used a statistical approach to predict mortality among patients who have been suffering with acute kidney injury \cite{Celi}. Herland \textit{et al.} have conducted a comprehensive survey on this topic, i.e., clinical data mining applications based on big data in health informatics \cite{Herland}. Cai \textit{et al.} have proposed an approach based on Baysian network to develop models using EHR for real time prediction of several targets, including the length of hospital stay, mortality, and readmission of hospitalized patients \cite{Cai:2016}. We note here that majority of the above studies focus on a single technique and mainly exploit data mining methods, whereas our work proposes a hybrid approach, features are selected at first and then ensemble models are generated using FVC technique with various ensemble classification techniques. Therefore it is believed that our current approach is applicable to any CDS related systems in general. 

\textcolor{blue}{Additionally, some of the above deep learning methodology based studies deal with Big data in the context of in-hospital mortality of ICU patients. According to Meyer \textit{et al.} \cite{Meyer2018} adding more variables into the actual data matrix increases the dimension of the matrix which may become computationally burdensome for this kind of models. But FVC technique \cite{mehedy-masud:2017:fvc} is reverse of this, deals with large data and has shown the improved performance on training or testing time.}

Finally, a subset of the authors were involved in proposing the FVC approach in \cite{mehedy-masud:2017:fvc} and ORCU (a novel orthogonal clustering technique) in \cite{mehedy-masud:2018:frmwrk} where some preliminary experiments and findings were discussed. In this work we proposed further improvements of the above mentioned works and conducted extensive experiments with real clinical data thereby reaching a promising milestone in this research endeavour.

%In 2015 Pontes \textit{et al.} has proposed bi-clustering \cite{Pontes:2015}. But it is treated that ORCU is a variation of bi-clustering.

%In 2016 Mohammad M. Masud and Abdel Rahman Al Harahsheh proposed ``feature vector compaction’’ (FVC) technique using lab test data for mortality prediction \cite{mehedy-masud:2017:fvc}. The study dealt with challenges associated with utilizing clinical test results and addressed few of them. Although it shows the FVC can improve the performance compared to the baseline but still it’s weighted score (i.e., F-score) are less than 78\% . In early 2018 Mohammad M. Masud and Muhsin Cheratta has introduced a framework for utilizing the lab test data for this clinical prediction task \cite{mehedy-masud:2018:frmwrk}. In this study they have proposed a novel orthogonal clustering, called ``ORCUE’’ technique to reduce data dimensions as well as missing data. Albeit of its improvement in weighted average score (i.e., F-score) in a lab test data still it suffers in generating a good result.  The highest score in this study is less than 70\%. There is lots of scope to improve the performance still demanding.


\section{Lab tests as features for MIMIC-II datasets} 
\includepdf[pages=-]{LAB-TESTS-DETAILS-MIMIC-II.pdf}

\section{Lab tests as features for MIMIC-III datasets} 
\includepdf[pages=-]{LAB-TESTS-DETAILS-MIMIC-III.pdf}

\section{Results in tabular format} 
\includepdf[pages=-]{Results-in-tabular-format.pdf}

\section{Top ten ranked features} 
\includepdf[pages=-]{Top-ten-ranked-features.pdf}

\section{Feature wise linear regression} 
\includepdf[pages=-]{Feature-wise-linear-reg.pdf}

	
	%% The Appendices part is started with the command \appendix;
	%% appendix sections are then done as normal sections
	%% \appendix
	
	%\appendix
%	\section{Orthogonal Clustering} \label{sec:appendices}
%	Appendix one text goes here.
%	
	% you can choose not to have a title for an appendix
	% if you want by leaving the argument blank
%	\section{Appendix two}
%	Appendix two text goes here.
		

	%% \section{}
	%% \label{}
	
	%% References
	%%
	%% Following citation commands can be used in the body text:
	%% Usage of \cite is as follows:
	%%   \cite{key}         ==>>  [#]
	%%   \cite[chap. 2]{key} ==>> [#, chap. 2]
	%%
	
	%% References with BibTeX database:
	
	%%%%%%%%%%%%%%%%%%%%%%%
	%% Elsevier bibliography styles
	%%%%%%%%%%%%%%%%%%%%%%%
	%% To change the style, put a % in front of the second line of the current style and
	%% remove the % from the second line of the style you would like to use.
	%%%%%%%%%%%%%%%%%%%%%%%
	
	%% Numbered
	%\bibliographystyle{model1-num-names}
	
	%% Numbered without titles
	%\bibliographystyle{model1a-num-names}
	
	%% Harvard
	%\bibliographystyle{model2-names.bst}\biboptions{authoryear}
	
	%% Vancouver numbered
	%\usepackage{numcompress}\bibliographystyle{model3-num-names}
	
	%% Vancouver name/year
	%\usepackage{numcompress}\bibliographystyle{model4-names}\biboptions{authoryear}
	
	%% APA style
	%\bibliographystyle{model5-names}\biboptions{authoryear}
	
	%% AMA style
	%\usepackage{numcompress}\bibliographystyle{model6-num-names}
	
	%% `Elsevier LaTeX' style
	\bibliographystyle{elsarticle-num}
	%%%%%%%%%%%%%%%%%%%%%%%
	\bibliography{mybibfile}
	
	%% Authors are advised to use a BibTeX database file for their reference list.
	%% The provided style file elsarticle-num.bst formats references in the required Procedia style
	
	%% For references without a BibTeX database:
	
	% \begin{thebibliography}{00}
	
	%% \bibitem must have the following form:
	%%   \bibitem{key}...
	%%
	
	% \bibitem{}
	
	% \end{thebibliography}
	
	%\input{experimenttables}
	
\end{document}

%%
%% End of file `ecrc-template.tex'. 